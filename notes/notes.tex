\documentclass[a4paper, 12pt]{article}
\usepackage[utf8]{inputenc}
\usepackage[T1]{fontenc}
\usepackage[ngerman,english]{babel}
\usepackage{tabularx}
\usepackage{booktabs}
\usepackage{amsmath, amsthm, amssymb}
\usepackage{siunitx}
\usepackage{caption}
\usepackage{graphicx}
\usepackage{hyperref}

\begin{document}

\section{Typical eye parameters}

These values are used as defaults for parameter optimization.

\begin{table}[h!]
    \centering
    \caption{Typical values for eye parameters}
    \begin{tabularx}{\textwidth}{l X r}
        \toprule
        Parameter             & Description                                                & Typical value       \\
        \midrule
        $R$                   & Corneal curvature radius                                   & \SI{7.8}{\mm}       \\
        $K$                   & Distance between pupil center and corneal curvature radius & \SI{4.2}{\mm}       \\
        $n_1$                 & Corneal refraction index                                   & \num{1.3375}        \\
        $\alpha_{\text{eye}}$ & Horizontal angle between visual and optical axis of eye    & \SI{\pm 5}{\degree} \\
        $\beta_{\text{eye}}$  & Vertical angle between visual and optical axis of eye      & \SI{1.5}{\degree}   \\
        \bottomrule
    \end{tabularx}
\end{table}

\section{Implementation}

\end{document}
